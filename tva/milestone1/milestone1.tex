\section{Milestone 1}
As suggested, the duration of milestone 1 is from the beginning of the project until the end of December.
Milestone 1 includes the set up of the software project which led to the discovery of the first bug.
It includes the first steps taken to gather information, get comfortable with the project and planning of the project.
Lastly, it includes the phase of testing the project regarding graph coverage.

\subsection{Set Up}\label{setup}
Instead of running natively on the developer's machine, EvaP is wrapped in a Vagrant execution environment%
\footnote{\url{https://www.vagrantup.com/}}.
This allows developers to develop features with their preferred tools on their favorite operating system out of Mac OS X, Windows, Debian and Centos, without having to go through the complicated process of collecting dependencies for their specific platform.
The complete project set up normally consists of installing git, a virtual machine provider for Vagrant and Vagrant itself, cloning the repository and running the shell command \texttt{vagrant up}.

As all core developers of EvaP use unix-based platforms, a bug in an external sub module used by EvaP remained unnoticed and undealt with:
When we tried to set up the project on windows machines that use a line feed and a carriage return (\texttt{LF}~\texttt{CR}) to end lines in text files --- as opposed to unix systems that only use a single line feed (\texttt{LF}) --- the set up failed.
The bug was fixed quickly: firstly only for EvaP, subsequently in the external module the line ending policy had to enforce \texttt{LF} only.

Now, the set up therefore works on all major operating systems, allowing an easy start for all developers who want to contribute to EvaP, and more insights into the system for us.

\subsection{First Steps}
Incidentally, this is the first semester for the student representatives to host \emph{EvaP Hacking Hours} biweekly.
This is an event to give students a space to develop and work on EvaP.
We will use it as a possibility to stay in contact with the main developers.
As testers of EvaP it is a valuable resource to be able to talk directly with developers.
This way we can gather current information easily.
We are able to verify the content of old artifacts with them as well as our future findings.

\subimport{/}{application_survey}

\subimport{/}{initial_test_plan}

\subimport{/}{test_automation}

\subsection{Graph Coverage}
\label{sec:graph-coverage}
According to our test plan we chose parts of EvaP to test based on graph coverage. 
We will test the function \texttt{send\_publish\_notifications} by creating a control flow graph and test cases based on path coverage criteria. 
We will investigate the documented finite state machine (FSM). 
We will create an activity diagram based on the documented use-case %TODO
Finally, we will investigate if our added test cases changed the line coverage, since this is the given measurement by the used tool \textit{Coveralls}.

\subimport{/}{control_flow_graph}

\subimport{/}{fsm}

\subimport{/}{use_case}

%TODO: something to add here? \subimport{/}{coveralls}
