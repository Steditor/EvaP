\section{Milestone 1}
As suggested, the duration of milestone 1 is from the beginning of the project until the end of December.
Milestone 1 includes the set up of the software project which led to the discovery of the first bug.
It includes the first steps taken to gather information, get comfortable with the project and planning of the project.
Lastly, it includes the phase of testing the project regarding graph coverage.

\subsection{Set Up}
- set up of vagrant on windows does not work for submodules
- unnoticed by developers because of different os


\subsection{First Steps}
Incidentally, this is the first semester for the student representatives to host \emph{EvaP Hacking Hours} biweekly.
This is an event to give students a space to develop and work on EvaP.
We will use it as a possibility to stay in contact with the main developers.
As testers of EvaP it is a valuable resource to be able to talk directly with developers.
This way we can gather current information easily.
We are able to verify the content of old artifacts with them as well as our future findings.

\subimport{/}{application_survey}

\subimport{/}{initial_test_plan}

\subimport{/}{test_automation}

\subsection{Graph Coverage}
According to our test plan we chose parts of EvaP to test based on graph coverage. 
We will test the function \texttt{send\_publish\_notifications} by creating a control flow graph and test cases based on path coverage criteria. 
We will investigate the documented finite state machine (FSM). 
We will create an activity diagram based on the documented use-case %TODO
Finally, we will investigate if our added test cases changed the line coverage, since this is the given measurement by the used tool COVERALLS.

\subimport{/}{control_flow_graph}

\subimport{/}{fsm}

\subimport{/}{use_case}

\subimport{/}{coveralls}
