\section{Introduction}

\subsection{EvaP -- An Evaluation System}
%History & Facts
The online platform EvaP is used for evaluation of courses at the Hasso Plattner Institue (HPI). 
Its development started in 2011 when the student representatives decided to redevelop the former system EvaJ.
Now, it is an Open Source project hosted on GitHub, even though the main developers are still part of the student representative team.

%Features
EvaP's time to shine is at the end of each semester. 
Each university course is evaluated with specific questionaires chosen by the lecturer. 
Students are allowed to give anonymous feedback to different aspects of the lecture, the lecturer itself and additional tutors through a grading system and comments.
EvaP encourages evaluation by rewarding participation with points. 
These reward points can be redeemed for currency to use at HPI events.
EvaP's latest feature is the distribution of course grades through the platform.

\subsection{Motivation -- Why Should EvaP Be Analyzed, Verified and Tested?}
%EvaP is important
EvaP is an important source for feedback at HPI. 
The platform offers students a way to express their critique anonymously.
It documents the feedback for lecturers.
Thus, they do not need to collect and save the feedback themselves.
Additionally, they can take their time to evaluate the student's feedback.
Furthermore, the evalutation of all courses is saved centrally.
This allows to gain an overview over the quality of HPI courses and compare feedback over time as well as with other courses.
Therefore, EvaP is an important tool at the HPI.

%Open Source + Student Development is complicated
Since EvaP is developed by students, the responsible persons and main developers change regularly as older students graduate and new ones enroll.
The change of responsible persons shifts the view of which features, programming paradigms and quality assurance are most important.
Consequently, the requirements change.
The change of main developers means a loss of knowledge about the existing code.
Besides, the Open Source aspect allows several developers without inside knowledge to add code.
Even though all code is reviewed and checked by the main developers, they may not grasp it as well as self-written code.
Events like Hackdays or Hacking Hours are used to promote EvaP's development.
While these practices ensure the advancement of EvaP, they may endanger its quality.

%Thus, EvaP needs to be tested
Because of EvaP's importance at the HPI, its quality should be ensured.
Therefore, we will analyze, verify and test the software within the scope of this lecture.
